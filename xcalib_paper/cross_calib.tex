\documentclass[12pt, preprint]{aastex}
%\usepackage{bm, graphicx, subfigure, amsmath, morefloats}
\bibliographystyle{apj}
\usepackage{caption}
\usepackage{subcaption}

% naming macros
\newcommand{\tc}{\textsl{The~Cannon}} 
\newcommand{\apogee}{\textsl{APOGEE}} 
\newcommand{\lamost}{\textsl{LAMOST}}
\newcommand{\segue}{\textsl{SEGUE}}
\newcommand{\rave}{\textsl{RAVE}}
\newcommand{\galah}{\textsl{GALAH}}
\newcommand{\gaiaeso}{\textsl{Gaia-ESO}}

% math and symbol macros
\newcommand{\teff}{\mbox{$\rm T_{eff}$}}
\newcommand{\feh}{\mbox{$\rm [Fe/H]$}}
\newcommand{\alphafe}{\mbox{$\rm [\alpha/Fe]$}}
\newcommand{\mh}{\mbox{$\rm [M/H]$}}
\newcommand{\logg}{\mbox{$\rm \log g$}}

\begin{document}

\title{Survey Cross-Calibration Using \tc: \\ \lamost\ Labels on the \apogee\ Scale}
\author{A. Y. Q. ~Ho\altaffilmark{1},
M.~Ness\altaffilmark{1},
David~W.~Hogg\altaffilmark{1,2,3}, 
H.-W.~Rix\altaffilmark{1},
M.~Fouesneau\altaffilmark{1},
C.~Liu\altaffilmark{4},
F.~Yang\altaffilmark{4}
}
\altaffiltext{1}{Max-Planck-Institut f\"ur Astronomie, K\"onigstuhl 17, D-69117 Heidelberg, Germany}
\altaffiltext{2}{Center for Cosmology and Particle Physics, Department of Phyics,
New York University, 4 Washington Pl., room 424, New York, NY, 10003, USA}
\altaffiltext{3}{Center for Data Science, New York University, 726 Broadway, 7th Floor, New York, NY 10003, USA}
\altaffiltext{4}{Key Laboratory of Optical Astronomy, National Astronomical Observatories, Chinese Academy of Sciences, Datun Road 20A,Beijin 100012, China}

\email{annaho@mpia.de}

\begin{abstract}

To capitalize on the new generation of large spectroscopic surveys, 
it is essential to develop robust methods for cross-calibration: for putting 
qualitatively surveys onto the same label (physical parameter and 
chemical abundance) scale. In this work, we demonstrate that
cross-calibration can be achieved using \tc\ \citep{ness2015}, a fast
new data-driven approach to stellar label determination, given a
comprehensive set of overlap objects between the surveys. 
For our demonstration, we apply \tc\ to 11,057 objects observed with 
both the \apogee\ DR12 and \lamost\ DR2 stellar surveys, and successfully
calculate four \apogee-scale labels (\teff, \logg, \feh, \alphafe) for all of
the \lamost\ spectra. We obtain values of
\teff, \logg, and \feh\ at least as ``good" (scatter and bias comparison here)
as those from the \lamost\ 
pipeline, with negligible bias from \apogee\ values (numbers??). In addition, we
successfully calculate the first reliable \alphafe\ values for \lamost\ 
spectra. 
The code used to produce the results as described in this paper was written in
Python and is available online for public 
use.\footnote{\url{www.github.com/annayqho/the-cannon}}

\end{abstract}

\keywords{
methods: data analysis
---
methods: statistical
---
stars: abundances
---
stars: fundamental parameters
---
surveys
---
techniques: spectroscopic
}

\section{Introduction: \tc\ in the Context of Survey Cross-Calibration}

A new generation of large-scale spectroscopic surveys (e.g. \segue\ \citep{Beers}, \rave\ \citep{Steinmetz2006}, \gaiaeso\ \citep{Gilmore2012}, \galah\ \citep{Freeman2012}, \apogee\ \citep{Majewski2012}, \lamost\ \citep{Newberg2012}) has been systematically 
measuring spectra for hundreds of thousands of stars in the Milky Way. In a few years,
this number will be in the millions. 
In the context of this unprecedented wealth of data, 
it is essential to develop tools that can measure consistent stellar labels 
(physical parameters and chemical abundances) from spectra: labels that
depend only on the star and not on survey-dependent properties such
as resolution and wavelength range. In other words, taking advantage of this data
necessitates developing robust techniques for cross-calibration. 

In this work, we set out to cross-calibrate the \lamost\ and \apogee\ surveys
using \tc. \tc\ is a data-driven method for measuring stellar labels in the context
of vast spectroscopic surveys. 
The method as applied to a single survey is described in detail in \citet{ness2015}; 
we recapitulate the fundamental assumptions and steps here, in the context of survey cross-calibration. 

\tc\ makes two assumptions: that the spectra of stars with identical labels look identical,
and that spectra vary smoothly with label changes. In other words, the continuum-normalized flux at each pixel in the spectrum is a smooth function of the labels. The function
that maps the flux in a pixel to the object's labels is called the ``spectral model." 

\tc\ proceeds in two steps: a ``training step" and then a ``test step." 
Presume that we would like to cross-calibrate Survey A and Survey B by measuring
labels for Survey A spectra that are on the Survey B scale.  
In the training step, \tc\ uses a set of ``reference objects" - a subset of spectra for
which corresponding stellar labels are ``known" and comprehensively
span the label space - to fit for the spectral model at each 
pixel independently. In the cross-calibration context, the training set would
consist of objects measured in common between the two surveys: training
spectra from Survey A and training labels from Survey B. 
The spectral model would thus map Survey A spectra to Survey B labels. 
In the test step, the spectral model is applied to each of the remaining objects
(known collectively as the ``test set") to infer their labels. 
In the cross-calibration context, the test set would consist of all the 
non-training spectra from Survey A. 
Because the model was trained on Survey B-scale labels, the inferred
labels will all be on the Survey B scale. 

The code used in this work is an implementation of the procedure outlined in 
\citep{ness2015}. A list of the (subtle) changes made since then
can be found in Appendix \ref{sec:appendix}.

\section{Data}

\subsection{\lamost\ Spectra}

Our goal is to calculate \apogee-scale labels for \lamost\ spectra: 
therefore, both the training and test objects consist of spectra from
\lamost\ DR2. We use 11,057 objects observed in common between
the two surveys: 803 for the training set and the rest for the test set. 

\lamost\ is blah blah. 

Before being fed to \tc, any spectroscopic data set must satisfy the criteria laid out in 
\citet{ness2015}. More specifically, the spectra must be
continuum-normalized in a consistent way that is independent of S/N, and need
to be radial velocity shifted and sampled onto a common wavelength grid with
a common line-spread function. Each pixel of each spectrum must be accompanied
by a flux variance that takes sources like photon noise into account. 

In order to satisfy these criteria, the displacement from the rest frame was
calculated for each \lamost\ spectrum using the redshift value provided in 
the header of each data file. The data was then resampled through interpolation
onto a common, evenly spaced grid of spacing 0.85\,$\mbox{\AA}$. 
Upper and lower cuts were applied to the data such that its lower and upper 
wavelengths were 3900\,$\mbox{\AA}$ and 8800\,$\mbox{\AA}$ respectively.
All of these operations were performed on the flux as well as inverse variance
arrays. 

\tc\ performs its own continuum normalization on raw spectra. 
Continuum normalization consists of identifying continuum pixels from continuum
normalized spectra, then applying a sinusoidal fit to those pixels in the raw spectra.
The training spectra are first pseudo continuum normalized using a running median: 
we take the 90th percentile from a running quantile window 400\,\AA\ wide. 
From the grid of pseudo continuum normalized training spectra, we create two vectors: 
the median flux across all the objects, and the variance in flux across all the objects. 
We then apply a median and flux cut such that 5\% of pixels in each 100\,\AA\ window
are selected as continuum. We then return to the raw spectra, and fit a third-order 
sinusoid to the continuum pixels. A separate sinusoid is fit for the red and blue spectral
wings. A sample training set spectrum with sinusoid continuum fit is shown in
Figure \ref{fig:sample_spec}. 

\begin{figure}[!htbp]
\centering
\includegraphics[scale=0.4]{poster_typical_spec_snr134_withcont.png}
\caption{A sample high-S/N LAMOST spectrum from the training set, 
overlaid with the continuum fit from \tc}
\label{fig:sample_spec}
\end{figure}

\subsection{\apogee\ Labels}

We choose to put \lamost\ spectra onto the \apogee\ scale instead of the
other way around because \apogee\ labels are presumed to be more precise
due to their derivation from higher resolution spectra. Thus, our model is
trained using labels from \apogee\ DR12. 

\apogee\ is part of the SDSS-III (Eisenstein et al. 2011). 
\apogee\ DR12 (Majewski et al. 2015) was released as part of the 12th data release of the Sloan
Digital Sky Survey (Alam et al. 2015, DR12). 

It is a high resolution
($R\approx22,500$) high SNR ($SNR\approx100$) H-band (15200-16900\,$\mbox{\AA}$) spectroscopic
survey of primarily red giants spanning the bulge, disk, and MW halo 
(Zasowski et al. 2013). \apogee's ASPCAP pipeline provides stellar labels for
these stars, which includes stellar parameters and multiple chemical 
abundances in addition to numerous flags that warn of problems with spectra
or fitting or label determination or both. It's based on chi-sq fitting of
the data to 1D LTE models for seven labels (Teff, logg, blah blah; Garcia 
Perez et al. 2015, in prep). 

\section{\tc\ Training Step: \\ Use Reference Objects to Fit a Spectral Model}

The two surveys have 11,061 objects in common. Of this overlap, a high 
fidelity sample was selected by applying three cuts to \apogee\ objects: 
first, a signal-to-noise cut such that objects with SNR\,\textgreater\,400 
were selected. Second, a velocity scatter cut such that objects with 
$\sigma_v\, < \,1$\,km/s were selected. And finally, only objects
without a BAD STAR flag were selected. With these cuts, 1722 stars were
selected to populate the training set with the remaining 9339 used as test
objects. 

\begin{figure}[!htbp]
\centering
\includegraphics[scale=0.4]{poster_first_order_coeffs.png}
\caption{Coefficients of the spectral model indicate how sensitive each pixel in the spectrum is to each of the labels. See that sensitive regions correspond to known spectral lines.}
\label{leading_coeffs}
\end{figure}

\section{\tc\ Test Step: \\ Use the Spectral Model to Infer Labels for Test Objects}

\begin{figure}
\centering
\includegraphics[scale=0.4]{poster_1to1_teff.png}
\end{figure}

\begin{figure}[!htbp]
\centering
\includegraphics[scale=0.4]{poster_1to1_teff_hist.png}
\end{figure}

\begin{figure}[!htbp]
\centering
\includegraphics[scale=0.4]{poster_1to1_logg.png}
\end{figure}

\begin{figure}[!htbp]
\centering
\includegraphics[scale=0.4]{poster_1to1_logg_hist.png}
\end{figure}

\begin{figure}[!htbp]
\centering
\includegraphics[scale=0.4]{poster_1to1_feh.png}
\end{figure}

\begin{figure}[!htbp]
\centering
\includegraphics[scale=0.4]{poster_1to1_feh_hist.png}
\end{figure}

\begin{figure}[!htbp]
\centering
\includegraphics[scale=0.4]{poster_1to1_alpha.png}
\end{figure}

\begin{figure}[!htbp]
\centering
\includegraphics[scale=0.4]{poster_1to1_alpha_hist.png}
\end{figure}

\begin{figure}[!htbp]
\centering
\includegraphics[scale=0.4]{poster_teff_logg_plane.png}
\end{figure}

\begin{figure}[!htbp]
\centering
\includegraphics[scale=0.4]{poster_feh_alpha_plane.png}
\end{figure}

\section{Discussion}

\tc\ works at low and high resolution! We put \lamost\ onto the \apogee\ 
parameter scale!

\section{Acknowledgements}

AYQH was partially supported by a Fulbright grant through the German-American
Fulbright Commission.

The research has received funding from the European Research Council under the 
European Union's Seventh Framework Programme (FP 7) ERC Grant Agreement n.
[321035].

\appendix
\addcontentsline{toc}{chapter}{APPENDIX}

\section{Appendix: Code Specifics}
\label{sec:appendix}

The following procedural changes were made to the code between the publication
of \citep{ness2015} and this cross-calibration work:

\begin{enumerate}
  \item \textbf{Continuum pixel identification:} instead of identifying
    continuum pixels through an iterative running quantile procedure, we 
    simply apply a median and variance cut to each pixel. More precisely,
    we assign a ``median flux'' and ``variance'' value to each pixel in the 
    spectrum, determined using the flux at that pixel across all of the 
    objects in the set. The median flux cut value was blah and the variance 
    cut value was blah. Both values were set in order to select 6\% of pixels 
    in each third of the spectrum. 
  \item Sines and cosines instead of Chebyshev polynomial
\end{enumerate}

\begin{thebibliography}{24}
\expandafter\ifx\csname natexlab\endcsname\relax\def\natexlab#1{#1}\fi

\bibitem[Beers et al.(2006)]{Beers} Beers, T.~C., Lee, Y., 
Sivarani, T., et al.\ 2006, \memsai, 77, 1171 

\bibitem[{{Freeman}(2012)}]{Freeman2012}
{Freeman}, K.~C. 2012, in Astronomical Society of the Pacific Conference
  Series, Vol. 458, Galactic Archaeology: Near-Field Cosmology and the
  Formation of the Milky Way, ed. W.~{Aoki}, M.~{Ishigaki}, T.~{Suda},
  T.~{Tsujimoto}, \& N.~{Arimoto}, 393

\bibitem[{{Gilmore} {et~al.}(2012){Gilmore}, {Randich}, {Asplund}, {Binney},
  {Bonifacio}, {Drew}, {Feltzing}, {Ferguson}, {Jeffries}, {Micela},
  {Negueruela}, {Prusti}, {Rix}, {Vallenari}, {Alfaro}, {Allende-Prieto},
  {Babusiaux}, {Bensby}, {Blomme}, {Bragaglia}, {Flaccomio}, {Fran{\c c}ois},
  {Irwin}, {Koposov}, {Korn}, {Lanzafame}, {Pancino}, {Paunzen},
  {Recio-Blanco}, {Sacco}, {Smiljanic}, {Van Eck}, \& {Walton}}]{Gilmore2012}
{Gilmore}, G., {Randich}, S., {Asplund}, M., {Binney}, J., {Bonifacio}, P.,
  {Drew}, J., {Feltzing}, S., {Ferguson}, A., {Jeffries}, R., {Micela}, G.,
  {Negueruela}, I., {Prusti}, T., {Rix}, H.-W., {Vallenari}, A., {Alfaro}, E.,
  {Allende-Prieto}, C., {Babusiaux}, C., {Bensby}, T., {Blomme}, R.,
  {Bragaglia}, A., {Flaccomio}, E., {Fran{\c c}ois}, P., {Irwin}, M.,
  {Koposov}, S., {Korn}, A., {Lanzafame}, A., {Pancino}, E., {Paunzen}, E.,
  {Recio-Blanco}, A., {Sacco}, G., {Smiljanic}, R., {Van Eck}, S., \& {Walton},
  N. 2012, The Messenger, 147, 25

\bibitem[{{Majewski}(2012)}]{Majewski2012}
{Majewski}, S.~R. 2012, in American Astronomical Society Meeting Abstracts,
  Vol. 219, American Astronomical Society Meeting Abstracts 219, 205.06

\bibitem[{{Ness}{ et~al.}(2015){Ness}, {Hogg}, {Rix}, {Ho}, \& 
  {Zasowski}}]{ness2015}
{Ness}, M., {Hogg}, D.W., {Rix}, H.-W., {Ho}, A.Y.Q., \& {Zasowski}, G. 2015

\bibitem[{{Newberg} {et~al.}(2012){Newberg}, {Carlin}, {Chen}, {Deng},
  {L{\'e}pine}, {Liu}, {Yang}, {Yuan}, {Zhang}, {Zhang}, {Legue Working Group},
  \& {Lamost-Plus Partnership}}]{Newberg2012}
{Newberg}, H.~J., {Carlin}, J.~L., {Chen}, L., {Deng}, L., {L{\'e}pine}, S.,
  {Liu}, X., {Yang}, F., {Yuan}, H.-B., {Zhang}, H., {Zhang}, Y., {Legue
  Working Group}, \& {Lamost-Plus Partnership}. 2012, in Astronomical Society
  of the Pacific Conference Series, Vol. 458, Galactic Archaeology: Near-Field
  Cosmology and the Formation of the Milky Way, ed. W.~{Aoki}, M.~{Ishigaki},
  T.~{Suda}, T.~{Tsujimoto}, \& N.~{Arimoto}, 405

\bibitem[{{Steinmetz} {et~al.}(2006){Steinmetz}, {Zwitter}, {Siebert},
  {Watson}, {Freeman}, {Munari}, {Campbell}, {Williams}, {Seabroke}, {Wyse},
  {Parker}, {Bienaym{\'e}}, {Roeser}, {Gibson}, {Gilmore}, {Grebel}, {Helmi},
  {Navarro}, {Burton}, {Cass}, {Dawe}, {Fiegert}, {Hartley}, {Russell},
  {Saunders}, {Enke}, {Bailin}, {Binney}, {Bland-Hawthorn}, {Boeche}, {Dehnen},
  {Eisenstein}, {Evans}, {Fiorucci}, {Fulbright}, {Gerhard}, {Jauregi}, {Kelz},
  {Mijovi{\'c}}, {Minchev}, {Parmentier}, {Pe{\~n}arrubia}, {Quillen}, {Read},
  {Ruchti}, {Scholz}, {Siviero}, {Smith}, {Sordo}, {Veltz}, {Vidrih}, {von
  Berlepsch}, {Boyle}, \& {Schilbach}}]{Steinmetz2006}
{Steinmetz}, M., {Zwitter}, T., {Siebert}, A., {Watson}, F.~G., {Freeman},
  K.~C., {Munari}, U., {Campbell}, R., {Williams}, M., {Seabroke}, G.~M.,
  {Wyse}, R.~F.~G., {Parker}, Q.~A., {Bienaym{\'e}}, O., {Roeser}, S.,
  {Gibson}, B.~K., {Gilmore}, G., {Grebel}, E.~K., {Helmi}, A., {Navarro},
  J.~F., {Burton}, D., {Cass}, C.~J.~P., {Dawe}, J.~A., {Fiegert}, K.,
  {Hartley}, M., {Russell}, K.~S., {Saunders}, W., {Enke}, H., {Bailin}, J.,
  {Binney}, J., {Bland-Hawthorn}, J., {Boeche}, C., {Dehnen}, W., {Eisenstein},
  D.~J., {Evans}, N.~W., {Fiorucci}, M., {Fulbright}, J.~P., {Gerhard}, O.,
  {Jauregi}, U., {Kelz}, A., {Mijovi{\'c}}, L., {Minchev}, I., {Parmentier},
  G., {Pe{\~n}arrubia}, J., {Quillen}, A.~C., {Read}, M.~A., {Ruchti}, G.,
  {Scholz}, R.-D., {Siviero}, A., {Smith}, M.~C., {Sordo}, R., {Veltz}, L.,
  {Vidrih}, S., {von Berlepsch}, R., {Boyle}, B.~J., \& {Schilbach}, E. 2006,
  \aj, 132, 1645

\bibitem[{{ Zasowski} {et~al.}(2013){Zasowski}, {Johnson}, {Frinchaboy},
  {Majewski}, {Nidever}, {Rocha Pinto}, {Girardi}, {Andrews}, {Chojnowski},
  {Cudworth}, {Jackson}, {Munn}, {Skrutskie}, {Beaton}, {Blake}, {Covey},
  {Deshpande}, {Epstein}, {Fabbian}, {Fleming}, {Garcia Hernandez}, {Herrero},
  {Mahadevan}, {M{\'e}sz{\'a}ros}, {Schultheis}, {Sellgren}, {Terrien}, {van
  Saders}, {Allende Prieto}, {Bizyaev}, {Burton}, {Cunha}, {da Costa},
  {Hasselquist}, {Hearty}, {Holtzman}, {Garc{\'{\i}}a P{\'e}rez}, {Maia},
  {O'Connell}, {O'Donnell}, {Pinsonneault}, {Santiago}, {Schiavon}, {Shetrone}, 
  {Smith}, \& {Wilson}}]{Zaso2013}
{Zasowski}, G., {Johnson}, J.~A., {Frinchaboy}, P.~M., {Majewski}, S.~R.,
  {Nidever}, D.~L., {Rocha Pinto}, H.~J., {Girardi}, L., {Andrews}, B.,
  {Chojnowski}, S.~D., {Cudworth}, K.~M., {Jackson}, K., {Munn}, J.,
  {Skrutskie}, M.~F., {Beaton}, R.~L., {Blake}, C.~H., {Covey}, K.,
  {Deshpande}, R., {Epstein}, C., {Fabbian}, D., {Fleming}, S.~W., {Garcia
  Hernandez}, D.~A., {Herrero}, A., {Mahadevan}, S., {M{\'e}sz{\'a}ros}, S.,
  {Schultheis}, M., {Sellgren}, K., {Terrien}, R., {van Saders}, J., {Allende
  Prieto}, C., {Bizyaev}, D., {Burton}, A., {Cunha}, K., {da Costa}, L.~N.,
  {Hasselquist}, S., {Hearty}, F., {Holtzman}, J., {Garc{\'{\i}}a P{\'e}rez},
  A.~E., {Maia}, M.~A.~G., {O'Connell}, R.~W., {O'Donnell}, C., {Pinsonneault},
  M., {Santiago}, B.~X., {Schiavon}, R.~P., {Shetrone}, M., {Smith}, V., \&
  {Wilson}, J.~C. 2013, \aj, 146, 81

\end{thebibliography}

\end{document}
