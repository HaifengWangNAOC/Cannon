\documentclass[12pt, preprint]{aastex}
%\usepackage{bm, graphicx, subfigure, amsmath, morefloats}
\bibliographystyle{apj}

% naming macros
\newcommand{\tc}{\textsl{The~Cannon}} 
\newcommand{\apogee}{\textsl{APOGEE}} 
\newcommand{\lamost}{\textsl{LAMOST}}

% math and symbol macros
\newcommand{\teff}{\mbox{$\rm T_{eff}$}}
\newcommand{\feh}{\mbox{$\rm [Fe/H]$}}
\newcommand{\alphafe}{\mbox{$\rm [\alpha/Fe]$}}
\newcommand{\mh}{\mbox{$\rm [M/H]$}}
\newcommand{\logg}{\mbox{$\rm \log g$}}

\begin{document}

\title{\lamost\ labels on the \apogee\ scale: \\ The first spectroscopic survey cross-calibration using \tc}
\author{A. Y. Q. ~Ho\altaffilmark{1},
M.~Ness\altaffilmark{1},
David~W.~Hogg\altaffilmark{1,2,3}, 
H.-W.~Rix\altaffilmark{1},
M.~Fouesneau\altaffilmark{1},
C.~Chao\altaffilmark{4},
F.~Yang\altaffilmark{4,5}
}
\altaffiltext{1}{Max-Planck-Institut f\"ur Astronomie, K\"onigstuhl 17, D-69117 Heidelberg, Germany}
\altaffiltext{2}{Center for Cosmology and Particle Physics, Department of Phyics,
New York University, 4 Washington Pl., room 424, New York, NY, 10003, USA}
\altaffiltext{3}{Center for Data Science, New York University, 726 Broadway, 7th Floor, New York, NY 10003, USA}
\altaffiltext{4}{Key Laboratory of Optical Astronomy, National Astronomical Observatories, Chinese Academy of Sciences, Datun Road 20A,Beijin 100012, China}
\altaffiltext{5}{University of Chinese Academy of Sciences, Beijing 100049, China}

\email{annaho@mpia.de}

\begin{abstract}

Using 12,000 objects observed with both the \apogee\ and \lamost\ stellar
surveys, we demonstrate the ability of \tc\ to bring qualitatively 
different stellar surveys onto a consistent stellar parameter and chemical 
abundance scale. 
\tc\ \citep{ness2015} is a data-driven method for determining stellar labels 
(physical parameters and chemical abundances) from stellar spectra in the 
context of vast spectroscopic surveys. 
Unlike many other methods for label determination, \tc\ does not 
explicitly rely on physics-based model spectra or line lists, and takes 
advantage of all of the pixels (and therefore information) in the spectrum. 
It works by fitting a flexible model that describes how the flux in each 
pixel of any given continuum-normalized spectrum in the survey depends on 
the labels. 
For this demonstration, we use SDSS/DR12 data from the 
SDSS-III/Apache Point Observatory Galactic Evolution Experiment (\apogee) 
as well as data from the \lamost second data release (DR2) to illustrate the 
success of \tc\ at different wavelength regions and resolution regimes. 
By training the model on \apogee\ spectra and using it to obtain labels for 
\lamost\ spectra, we obtain labels that are consistent
with the \apogee\ labels for these objects, thus successfully putting spectra
from \lamost\ onto the \apogee\ parameter scale. 
The code used to generate the results in this paper is available for use by
the community and can be found at this link (or downloaded in this way?)

\end{abstract}

\keywords{
methods: data analysis
---
methods: statistical
---
stars: abundances
---
stars: fundamental parameters
---
surveys
---
techniques: spectroscopic
}

\section{Introduction}

\tc\ was conceived with the goal of consistently measuring stellar labels 
(physical parameters and chemical abundances) for the hundreds of thousands of 
stars in the Milky Way observed with state-of-the-art spectroscopic surveys. 
The method is described in detail in \citep{ness2015} and we recapitulate the 
two fundamental steps here. First, \tc\ uses a set of \emph{reference objects}
in a survey to create a model. Reference objects are a subset of spectra for
which corresponding stellar labels are known with high fidelity, for
calibration reasons or otherwise. These objects are used as a
\emph{training set} to fix a flexible generative model that describes 
a probability density function for the flux in each pixel of a 
continuum-normalized spectrum as a function of the ``known'' labels.
In the second step, the model is assumes to hold for all the 
other objects in the survey (known collectively as the \emph{test set}) and 
is applied to infer their labels. 

One of the limitations of this method is its reliance on the existence of a 
training set that is both trustworthy and fully spans the label space.
Major strengths of this method include speed, independence from physical 
models and line lists, and weak degradation with decreasing signal-to-noise. 
Perhaps most importantly, the method has the potential for bringing 
qualitatively different surveys onto a consistent physical scale.

To perform this cross-calibration, we use \apogee\ DR12 spectra as the 
training set and \lamost\ DR2 spectra as the test set, 
thus bringing \lamost\ spectra onto the \apogee\ scale. 
Cross-calibrating these two surveys provides an excellent opportunity to test
the performance of \tc\ in both different wavelength and different resolution
regimes. \apogee\ wavelength: H-band (15200-16900 Angstroms) at resolution 
R ~ 22,500. \lamost\ DR2: R ~ 1,800. 370-900nm. 

The code used can be found here. This code is an implementation of the 
procedure outlined in \citep{ness2015}. However, there are some differences
from this original paper and those differences can be foudn in the appendix.

\section{Preparing for \tc: Selection and Pre-Processing of Data}

\apogee\ DR12 data overview: SDSS-III/\apogee\ survey. APOGEE-1 survey 
(Majewski et al. 2015) released as part of the 12th data release of the Sloan
Digital Sky Survey (Alam et al. 2015, DR12). The pre-continuum normalized
spectra have gone through the APOGEE data reduction pipeline 
(Nidever et al. 2015). APOGEE survey (Majewski et al. 2012, 2015 in prep). 
APOGEE, part of the SDSS-III (Eisenstein et al. 2011) is a high-res
(R~22,500) high SNR (SNR~100) H-band (15200-16900 Angstroms) spectroscpoic
survey of primarily red giants spanning the bulge, disk, and MW halo 
(Zasowski et al. 2013). APOGEE'S ASPCAP pipeline provides stellar labels for
these stars, which includes stellar parameters and multiple chemical 
abundances in addition to numerous flags that warn of problems with spectra
or fitting or label determination or both. It's based on chi-sq fitting of
the data to 1D LTE models for seven labels (Teff, logg, blah blah; Garcia 
Perez et al. 2015, in prep). 
\subsection{Choosing a Training Set for Cross-Calibration}

All overlap, then select some fraction of the highest signal-to-noise (S/N) 
stars. But need to represent and cover label space as well. Repeat with 
various fractions of the 12,000 stars in common.

\subsection{Data Pre-Processing Steps}

Before being fed to \tc, any spectroscopic data set (whether used as the 
training or the test set) must satisfy the criteria laid out in 
\citep{ness2015}. More specifically, the spectra must be
continuum-normalized in a consistent way that is independent of S/N, and need
to be radial velocity shifted and sampled onto a common wavelength grid with
a common line-spread function. Each pixel of each spectrum must be accompanied
by a flux variance that takes sources like photon noise into account. 

These are the criteria that apply to any data set. We now illustrate the 
preparation of two different data sets for use by \tc: the \apogee\ training
set and the \lamost\ test set. 

\subsubsection{\apogee\ Data Pre-Processing}

Similar to what's in the original method paper.

\subsubsection{\lamost\ Data Pre-Processing}

Details from Melissa's e-mail.

\section{\tc}


\subsection{Continuum Normalization}

After data has been shaped up and prepared for use by \tc\, \tc\ does its own
continuum normalization. Find a certain fraction of pixels to use as continuum
Fit a series of sines and cosines. Divide the spectra by this function. This 
is different to what we did in the original \tc\ method paper \citep{ness2015}. 


\section{Results: Stellar Labels for \lamost}

\subsection{Take-One-Out Tests}

\section{Discussion}

\tc\ works at low and high resolution! We put \lamost\ onto the \apogee\ 
parameter scale!

\section{Acknowledgements}

AYQH was partially supported by a Fulbright grant through the German-American
Fulbright Commission.

The research has received funding from the European Research Council under the 
European Union's Seventh Framework Programme (FP 7) ERC Grant Agreement n.
[321035].

\appendix
\addcontentsline{toc}{chapter}{APPENDIX}

\section{Appendix: Code Specifics}
Specific differences about the code.

\begin{thebibliography}{24}
\expandafter\ifx\csname natexlab\endcsname\relax\def\natexlab#1{#1}\fi

\bibitem[{{Ness}{ et~al.}(2015){Ness}, {Hogg}, {Rix}, {Ho}, \& 
  {Zasowski}}]{ness2015}
{Ness}, M., {Hogg}, D.W., {Rix}, H.-W., {Ho}, A.Y.Q., \& {Zasowski}, G. 2015

\bibitem[{{ Zasowski} {et~al.}(2013){Zasowski}, {Johnson}, {Frinchaboy},
  {Majewski}, {Nidever}, {Rocha Pinto}, {Girardi}, {Andrews}, {Chojnowski},
  {Cudworth}, {Jackson}, {Munn}, {Skrutskie}, {Beaton}, {Blake}, {Covey},
  {Deshpande}, {Epstein}, {Fabbian}, {Fleming}, {Garcia Hernandez}, {Herrero},
  {Mahadevan}, {M{\'e}sz{\'a}ros}, {Schultheis}, {Sellgren}, {Terrien}, {van
  Saders}, {Allende Prieto}, {Bizyaev}, {Burton}, {Cunha}, {da Costa},
  {Hasselquist}, {Hearty}, {Holtzman}, {Garc{\'{\i}}a P{\'e}rez}, {Maia},
  {O'Connell}, {O'Donnell}, {Pinsonneault}, {Santiago}, {Schiavon}, {Shetrone}, 
  {Smith}, \& {Wilson}}]{Zaso2013}
{Zasowski}, G., {Johnson}, J.~A., {Frinchaboy}, P.~M., {Majewski}, S.~R.,
  {Nidever}, D.~L., {Rocha Pinto}, H.~J., {Girardi}, L., {Andrews}, B.,
  {Chojnowski}, S.~D., {Cudworth}, K.~M., {Jackson}, K., {Munn}, J.,
  {Skrutskie}, M.~F., {Beaton}, R.~L., {Blake}, C.~H., {Covey}, K.,
  {Deshpande}, R., {Epstein}, C., {Fabbian}, D., {Fleming}, S.~W., {Garcia
  Hernandez}, D.~A., {Herrero}, A., {Mahadevan}, S., {M{\'e}sz{\'a}ros}, S.,
  {Schultheis}, M., {Sellgren}, K., {Terrien}, R., {van Saders}, J., {Allende
  Prieto}, C., {Bizyaev}, D., {Burton}, A., {Cunha}, K., {da Costa}, L.~N.,
  {Hasselquist}, S., {Hearty}, F., {Holtzman}, J., {Garc{\'{\i}}a P{\'e}rez},
  A.~E., {Maia}, M.~A.~G., {O'Connell}, R.~W., {O'Donnell}, C., {Pinsonneault},
  M., {Santiago}, B.~X., {Schiavon}, R.~P., {Shetrone}, M., {Smith}, V., \&
  {Wilson}, J.~C. 2013, \aj, 146, 81

\end{thebibliography}

\end{document}
