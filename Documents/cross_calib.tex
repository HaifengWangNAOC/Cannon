\documentclass[12pt, preprint]{aastex}
%\usepackage{bm, graphicx, subfigure, amsmath, morefloats}
\bibliographystyle{apj}

% naming macros
\newcommand{\tc}{\textsl{The~Cannon}} 
\newcommand{\apogee}{\textsl{APOGEE}} 
\newcommand{\lamost}{\textsl{LAMOST}}

% math and symbol macros
\newcommand{\teff}{\mbox{$\rm T_{eff}$}}
\newcommand{\feh}{\mbox{$\rm [Fe/H]$}}
\newcommand{\alphafe}{\mbox{$\rm [\alpha/Fe]$}}
\newcommand{\mh}{\mbox{$\rm [M/H]$}}
\newcommand{\logg}{\mbox{$\rm \log g$}}

\begin{document}

\title{\lamost\ labels on the \apogee\ scale: \\ The first spectroscopic survey cross-calibration using \tc}
\author{A. Y. Q. ~Ho\altaffilmark{1},
M.~Ness\altaffilmark{1},
David~W.~Hogg\altaffilmark{1,2,3}, 
H.-W.~Rix\altaffilmark{1},
M.~Fouesneau\altaffilmark{1},
C.~Chao\altaffilmark{4},
F.~Yang\altaffilmark{4,5}
}
\altaffiltext{1}{Max-Planck-Institut f\"ur Astronomie, K\"onigstuhl 17, D-69117 Heidelberg, Germany}
\altaffiltext{2}{Center for Cosmology and Particle Physics, Department of Phyics,
New York University, 4 Washington Pl., room 424, New York, NY, 10003, USA}
\altaffiltext{3}{Center for Data Science, New York University, 726 Broadway, 7th Floor, New York, NY 10003, USA}
\altaffiltext{4}{Key Laboratory of Optical Astronomy, National Astronomical Observatories, Chinese Academy of Sciences, Datun Road 20A,Beijin 100012, China}
\altaffiltext{5}{University of Chinese Academy of Sciences, Beijing 100049, China}

\email{annaho@mpia.de}

\begin{abstract}

Using 12,000 objects observed with both the \apogee\ and \lamost\ stellar
surveys, we demonstrate the ability of \tc\ to bring qualitatively 
different stellar surveys onto a consistent stellar parameter and chemical 
abundance scale. 
\tc\ \citep{ness2015} is a data-driven method for determining stellar labels 
(physical parameters and chemical abundances) from stellar spectra in the 
context of vast spectroscopic surveys. 
Unlike many other methods for label determination, \tc\ does not 
explicitly rely on physics-based model spectra or line lists, and takes 
advantage of all of the pixels (and therefore information) in the spectrum. 
It works by fitting a flexible model that describes how the flux in each 
pixel of any given continuum-normalized spectrum in the survey depends on 
the labels. 
For this demonstration, we train the model on \apogee\ spectra and use it
to obtain labels for \lamost\ spectra. We obtain labels that are consistent
with the \apogee\ labels for these objects, thus successfully putting spectra
from \lamost\ onto the \apogee\ parameter scale. 

\end{abstract}

\keywords{
methods: data analysis
---
methods: statistical
---
stars: abundances
---
stars: fundamental parameters
---
surveys
---
techniques: spectroscopic
}

\section{Introduction}

Reference the original Cannon paper \citep{ness2015}. Say that if people want
to cite \tc\ in general that they should cite both of these papers. Give an 
overview of \apogee\ data as well as an overview of \lamost\ data. Point out
that these two surveys are at different wavelength regions and different 
resolutions, providing a great opportunity for a first test of the performance
of \tc\ and its potential for cross-calibrating surveys and putting labels
onto the same scale.

\section{Preparing Data for \tc}

Data preparation for \tc\ in general. Needs to be radial velocity shifted,
be sampled onto a common wavelength grid, need to be provided with inverse
variance estimates (one per pixel), common line-spreaad function, some kind of
consistent continuum normalization based on fitting or means but not medians
or quantiles. This is because we don't want the model to be distorted as a 
function of SNR. Next we go into the specifics of pre-processing for \apogee\
and for \lamost\ as an illustration of how data of different formats needs to 
be handled prior to use by \tc.

\subsection{\apogee\ Data Pre-Processing}

Similar to what's in the original method paper.

\subsection{\lamost\ Data Pre-Processing}

Details from Melissa's e-mail.

\section{\tc}

\subsection{Continuum Normalization}

After data has been shaped up and prepared for use by \tc\, \tc\ does its own
continuum normalization. Find a certain fraction of pixels to use as continuum
Fit a series of sines and cosines. Divide the spectra by this function. This 
is different to what we did in the original \tc\ method paper \citep{ness2015}. 

\subsection{Choosing a Training Set for Cross-Calibration}

All overlap, then select some fraction of the highest signal-to-noise (S/N) 
stars. But need to represent and cover label space as well. Repeat with 
various fractions of the 12,000 stars in common.

\section{Results: Stellar Labels for \lamost}

\subsection{Take-One-Out Tests}

\section{Discussion}

\tc\ works at low and high resolution! We put \lamost\ onto the \apogee\ 
parameter scale!

\section{Acknowledgements}

\begin{thebibliography}{24}
\expandafter\ifx\csname natexlab\endcsname\relax\def\natexlab#1{#1}\fi

\bibitem[{{Ness}{ et~al.}(2015){Ness}, {Hogg}, {Rix}, {Ho}, \& 
  {Zasowski}}]{ness2015}
{Ness}, M., {Hogg}, D.W., {Rix}, H.-W., {Ho}, A.Y.Q., \& {Zasowski}, G. 2015

\bibitem[{{ Zasowski} {et~al.}(2013){Zasowski}, {Johnson}, {Frinchaboy},
  {Majewski}, {Nidever}, {Rocha Pinto}, {Girardi}, {Andrews}, {Chojnowski},
  {Cudworth}, {Jackson}, {Munn}, {Skrutskie}, {Beaton}, {Blake}, {Covey},
  {Deshpande}, {Epstein}, {Fabbian}, {Fleming}, {Garcia Hernandez}, {Herrero},
  {Mahadevan}, {M{\'e}sz{\'a}ros}, {Schultheis}, {Sellgren}, {Terrien}, {van
  Saders}, {Allende Prieto}, {Bizyaev}, {Burton}, {Cunha}, {da Costa},
  {Hasselquist}, {Hearty}, {Holtzman}, {Garc{\'{\i}}a P{\'e}rez}, {Maia},
  {O'Connell}, {O'Donnell}, {Pinsonneault}, {Santiago}, {Schiavon}, {Shetrone}, 
  {Smith}, \& {Wilson}}]{Zaso2013}
{Zasowski}, G., {Johnson}, J.~A., {Frinchaboy}, P.~M., {Majewski}, S.~R.,
  {Nidever}, D.~L., {Rocha Pinto}, H.~J., {Girardi}, L., {Andrews}, B.,
  {Chojnowski}, S.~D., {Cudworth}, K.~M., {Jackson}, K., {Munn}, J.,
  {Skrutskie}, M.~F., {Beaton}, R.~L., {Blake}, C.~H., {Covey}, K.,
  {Deshpande}, R., {Epstein}, C., {Fabbian}, D., {Fleming}, S.~W., {Garcia
  Hernandez}, D.~A., {Herrero}, A., {Mahadevan}, S., {M{\'e}sz{\'a}ros}, S.,
  {Schultheis}, M., {Sellgren}, K., {Terrien}, R., {van Saders}, J., {Allende
  Prieto}, C., {Bizyaev}, D., {Burton}, A., {Cunha}, K., {da Costa}, L.~N.,
  {Hasselquist}, S., {Hearty}, F., {Holtzman}, J., {Garc{\'{\i}}a P{\'e}rez},
  A.~E., {Maia}, M.~A.~G., {O'Connell}, R.~W., {O'Donnell}, C., {Pinsonneault},
  M., {Santiago}, B.~X., {Schiavon}, R.~P., {Shetrone}, M., {Smith}, V., \&
  {Wilson}, J.~C. 2013, \aj, 146, 81

\end{thebibliography}

\end{document}
